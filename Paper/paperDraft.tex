\documentclass[a4paper,fleqn,11pt]{article}

\usepackage[margin=25mm]{geometry}

\listfiles
\newcommand{\bx}[1]{\fbox{\begin{minipage}{12cm}#1\end{minipage}}}
\usepackage{url,apalike,amsmath}
\usepackage[pdftex]{graphicx}
\usepackage[small]{titlesec}

\begin{document}

\large\noindent\textbf{Searching for Unknotting Sequences using
  Machine Learning}\\
\\
%% authorlist
\normalsize

\medskip
\hrule
\medskip

%%%%%%%%%%%%%%%%%%%%%%%%%%%%%%%%%%%%%%%%%%%%%%%%%%%%%%%%%%%%%%%%%%%%%%%%%%
\section{Introduction}

\bx{motivation - why are we doing this? Perhaps contextualise this in
  the idea of different kinds of computational tools for mathematics
  research: we are accustomed to tools that 

%%%%%%%%%%%%%%%%%%%%%%%%%%%%%%%%%%%%%%%%%%%%%%%%%%%%%%%%%%%%%%%%%%%%%%%%%%
\section{Problem Description}

\bx{describe problem in detail. give details of what each of the
  operators does}


%%%%%%%%%%%%%%%%%%%%%%%%%%%%%%%%%%%%%%%%%%%%%%%%%%%%%%%%%%%%%%%%%%%%%%%%%%
\section{Algorithm}

Genetic algorithms (GAs) are a population-based search heuristic,
designed to efficiently search a large space of possible solutions to
a problem. They are inspired by 


\bx{a description of the algorithm. I assume that we are going to use
  the SSM method? Describe the representation of the operator list,
  and the fitness function - and, in particular, how that fitness
  function is designed. A pseudocode version of the search process
  would be useful. We are looking just at finding the sequence for
  unknotting an individual knot, not a general sequence for unknotting
  a number of different knots, I assume?}

%%%%%%%%%%%%%%%%%%%%%%%%%%%%%%%%%%%%%%%%%%%%%%%%%%%%%%%%%%%%%%%%%%%%%%%%%%
\section{Experimental Setup and Results}

\bx{describe the experimental setup. how many attempts, how big is the
  knot, how many times did we repeat the algorithmic run.}

\bx{Present a table of results. We need to run each experiment a
  number of times (because it is a stochastic algorithm) for a number
  of knots (so that a pathological case doesn't distort the
  results). Each experiment is a particular size of knot.}

%%%%%%%%%%%%%%%%%%%%%%%%%%%%%%%%%%%%%%%%%%%%%%%%%%%%%%%%%%%%%%%%%%%%%%%%%%
\section{Discussion}

\bx{...of results}

%%%%%%%%%%%%%%%%%%%%%%%%%%%%%%%%%%%%%%%%%%%%%%%%%%%%%%%%%%%%%%%%%%%%%%%%%%
\bibliographystyle{plain}
\bibliography{refs.bib}

\end{document}
